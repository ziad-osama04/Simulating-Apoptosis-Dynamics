\documentclass[conference]{IEEEtran}
\IEEEoverridecommandlockouts
\usepackage{cite}
\usepackage{amsmath,amssymb,amsfonts}
\usepackage{graphicx}
\usepackage{textcomp}
\usepackage{xcolor}
\usepackage{booktabs}

\def\BibTeX{{\rm B\kern-.05em{\sc i\kern-.025em b}\kern-.08em
    T\kern-.1667em\lower.7ex\hbox{E}\kern-.125emX}}

\begin{document}

\title{Simulating the Dynamics of Programmed Cell Death}

\author{
    \IEEEauthorblockN{1\textsuperscript{st} Ziad Osama}
    \IEEEauthorblockA{\textit{Faculty of Engineering} \\
        \textit{Cairo University}\\
        ziad.ebrahim04@eng-st.cu.edu.eg}
    \and
    \IEEEauthorblockN{2\textsuperscript{nd} Anas Mohamed}
    \IEEEauthorblockA{\textit{Faculty of Engineering} \\
        \textit{Cairo University}\\
       anas.bayoumi05@eng-st.cu.edu.eg}
    \and
    \IEEEauthorblockN{3\textsuperscript{rd} Ahmed Mahmoud}
    \IEEEauthorblockA{\textit{Faculty of Engineering} \\
        \textit{Cairo University}\\
        ahmed.abdelzaher04@eng-st.cu.edu.eg}
    \and
    \IEEEauthorblockN{4\textsuperscript{th} Hassan Badawy}
    \IEEEauthorblockA{\textit{Faculty of Engineering} \\
        \textit{Cairo University}\\
        hassan.badawy05@eng-st.cu.edu.eg}
    \and
    \IEEEauthorblockN{5\textsuperscript{th} Mohamed Ehab}
    \IEEEauthorblockA{\textit{Faculty of Engineering} \\
        \textit{Cairo University}\\
       mohamed.mohamed056@eng-st.cu.edu.eg}
    \and
    \IEEEauthorblockN{6\textsuperscript{th} Menna Atef}
    \IEEEauthorblockA{\textit{Faculty of Engineering} \\
        \textit{Cairo University}\\
       menna.mahmoud06@eng-st.cu.edu.eg}
    \and
    \IEEEauthorblockN{7\textsuperscript{th} Engy Mohamed}
    \IEEEauthorblockA{\textit{Faculty of Engineering} \\
        \textit{Cairo University}\\
        engy.elsarta05@eng-st.cu.edu.eg}
    \and
    \IEEEauthorblockN{8\textsuperscript{th} Nada Mostafa}
    \IEEEauthorblockA{\textit{Faculty of Engineering} \\
        \textit{Cairo University}\\
        Nada.Morad05@eng-st.cu.edu.eg}
    \and
    \IEEEauthorblockN{9\textsuperscript{th} Saga Sadek}
    \IEEEauthorblockA{\textit{Faculty of Engineering} \\
        \textit{Cairo University}\\
        saga.soliman05@eng-st.cu.edu.eg}
}

\maketitle

\begin{abstract}
Accurately simulating complex biological processes like apoptosis is crucial for advancing biomedical research. In this work, we conduct a comprehensive evaluation of numerical methods for solving ordinary differential equations (ODEs), with a particular emphasis on comparing fixed-step and adaptive step-size approaches. Specifically, we develop and implement three classical solvers from scratch: the first-order Euler method, the fourth-order Runge-Kutta (RK4) method, and the adaptive Runge-Kutta-Fehlberg (RKF45) method. These algorithms are applied to a six-dimensional ODE model representing the dynamics of apoptosis. The adaptive RKF45 method, known for its high accuracy, is used as a reference to quantitatively assess the performance of the fixed-step methods. Our findings demonstrate the advantages of adaptive step-size control in terms of efficiency and reliability, while also providing a clear discussion of the trade-offs between computational cost, accuracy, and algorithmic complexity.

Furthermore, we explore a modern machine learning approach by employing Physics-Informed Neural Networks (PINNs) to solve the same ODE system, and we compare its results to those obtained from traditional numerical solvers.

Keywords: Apoptosis, ODE, Euler Method, Runge-Kutta, Numerical Analysis, Systems Biology, Physics-Informed Neural Networks, Machine Learning.
\end{abstract}

\section{Introduction}
Apoptosis, or programmed cell death, is a crucial biological process that maintains cellular health by eliminating damaged or unnecessary cells. Disruptions in apoptosis can lead to diseases such as cancer or neurodegeneration. Mathematical models of apoptosis, typically formulated as systems of ordinary differential equations (ODE), are crucial to understanding the underlying cellular mechanisms \cite{bib:laise}. The choice of numerical solver to integrate these equations profoundly impacts the simulation's accuracy and efficiency. Although fixed-step methods are foundational, they can be inefficient. They must use a single small step size throughout the entire simulation, even in regions where the solution is changing slowly.

This paper explores a more advanced approach by implementing and comparing four distinct numerical solvers.
\begin{enumerate}
    \item The \textbf{Runge-Kutta-Fehlberg (RKF45) method}, a modern adaptive step solver that will serve as our high accuracy baseline.
    \item The fixed-step \textbf{ fourth order Runge-Kutta (RK4) method}.
    \item The fixed-step \textbf{ explicit Euler method}.
    \item A \textbf{Physics-Informed Neural Network (PINN)} machine learning approach.
\end{enumerate}
By implementing all four from scratch, we provide a clear demonstration of their underlying logic and a quantitative comparison of their performance.

\section{Literature Review}
The computational study of apoptosis typically relies on solving Ordinary Differential Equation (ODE) models, such as the one by Laise et al. The choice of numerical solver is critical. Foundational fixed-step methods include the simple first-order Euler method and the highly accurate fourth-order Runge-Kutta (RK4) method. A major advancement was the development of adaptive step-size solvers like the Runge-Kutta-Fehlberg (RKF45), which dynamically adjust the step size to control error, ensuring both efficiency and reliability. More recently, machine learning approaches like Physics-Informed Neural Networks (PINNs) have emerged as a powerful, mesh-free alternative for solving ODEs and tackling inverse problems in systems biology. This paper provides a direct performance benchmark of the foundational numerical methods (Euler, RK4, and RKF45) as well as a PINN-based approach that is central to this evolving field.

\section{Methodology}
\subsection{ODE Model of Apoptosis}
The apoptosis model consists of six coupled nonlinear ODEs that describe the concentration dynamics of hypoxia-inducible factor ($y_{hif}$), oxygen ($y_{o2}$), coactivator p300 ($y_{p300}$), tumor suppressor p53 ($y_{p53}$), caspase ($y_{casp}$), and potassium ions ($y_{kp}$):

\begin{align}
\frac{dy_{hif}}{dt} &= a_{hif} - a_3 y_{o2} y_{hif} - a_4 y_{hif} y_{p300} - a_7 y_{p53} y_{hif} \\
\frac{dy_{o2}}{dt} &= a_{o2} - a_3 y_{o2} y_{hif} + a_{11} y_{hif} y_{p300} - a_{11} y_{o2} \\
\frac{dy_{p300}}{dt} &= -a_4 y_{hif} y_{p300} - a_5 y_{p300} y_{p53} + a_8 \\
\frac{dy_{p53}}{dt} &= a_{p53} - a_5 y_{p300} y_{p53} - a_9 y_{p53} \\
\frac{dy_{casp}}{dt} &= a_9 y_{p53} + a_{12} - a_{13} y_{casp} \\
\frac{dy_{kp}}{dt} &= -a_{10} y_{casp} y_{kp} + a_{11} y_{o2} - a_{14} y_{kp}
\end{align}

This system captures the biochemical regulation of apoptosis, showing how oxygen and genetic regulators control caspase activation and potassium ion loss—hallmarks of programmed cell death.

\subsection{Parameter Settings}
All simulations are run over a time interval of $t \in [0, 100]$. The system is solved using the initial condition $\mathbf{y_0} = [1, 0, 0, 0, 0, 0]$, which represents a hypoxic state where only the HIF-1 factor is initially active. The model parameters that govern the reaction rates are adopted from the original study and are as follows (all values in appropriate units):

\begin{center}
\begin{tabular}{llllll}
$a_{hif} = 1.52$ & $a_{o2} = 1.8$ & $a_{p53} = 0.05$ & $a_3 = 0.9$  \\
$a_7 = 0.7$ & $a_8 = 0.06$ & $a_9 = 0.1$ & $a_{10} = 0.7$  \\
$a_4 = 0.2$ & $a_5 = 0.001$ & $a_{11} = 0.2$ & $a_{12} = 0.1$ \\
$a_{13} = 0.1$ & $a_{14} = 0.05$ \\
\end{tabular}
\end{center}

The numerical methods used to solve this system are detailed in the subsequent sections.

\section{The Baseline Solver: Runge-Kutta-Fehlberg (RKF45) Method [3][5]}
\subsection{Logic and Concept}
To establish a high-accuracy baseline solution, we implement the adaptive Runge-Kutta-Fehlberg (RKF45) method. RKF45 is an embedded Runge-Kutta method, meaning at each step it calculates two different approximations of the next point: a 4th-order estimate ($\mathbf{y}^{(4)}$) and a more accurate 5th-order estimate ($\mathbf{y}^{(5)}$). This is achieved efficiently using a shared set of six function evaluations. The difference between these two results provides a direct estimate of the local truncation error ($\mathbf{ee} = ||\mathbf{y}^{(5)} - \mathbf{y}^{(4)}||$). This error is then compared to a user-defined tolerance. If the error is too large, the step is rejected and re-calculated with a smaller step size, $h$. If the error is acceptable, the step is taken, and $h$ may be increased for the next step. This adaptive control ensures both accuracy and efficiency.

\subsection{Results}
\begin{figure}[htp]
\centerline{\includegraphics[width=\columnwidth]{rkf45_results.png}}
\caption{Model solution using RKF45 Method.}
\label{fig:rkf45}
\end{figure}

\section{The 4th-Order Runge-Kutta (RK4) Method[2]}
\subsection{Logic and Concept}
To achieve a higher degree of accuracy, we implement the 4th-Order Runge-Kutta (RK4) method. This is an explicit, single-step solver that significantly improves upon the Euler method by sampling the derivative (the slope) at four strategic points within each time step. These points include the beginning, two distinct estimates at the midpoint, and the end of the interval.

By computing a weighted average of these four slope estimates, the RK4 method creates a much more accurate projection to the next point. This process effectively corrects for the curvature of the solution path, which is a major source of error in first-order methods. The iterative formula is given by:
\begin{equation}
    \mathbf{y}_{n+1} = \mathbf{y}_n + \frac{h}{6}(k_1 + 2k_2 + 2k_3 + k_4)
\end{equation}
where $h$ is the step size and the terms $k_1, k_2, k_3,$ and $k_4$ are the four calculated slope estimates. The method's global error is proportional to $h^4$, providing a substantial gain in accuracy and making it a standard choice for solving non-stiff ODE problems.

\subsection{Results}
\begin{figure}[htbp]
\centerline{\includegraphics[width=\columnwidth]{RK4.png}}
\caption{Model solution using Runge-Kutta Method.}
\label{fig:rk4}
\end{figure}
\
\
\
\section{The Explicit Euler Method[1]}
\subsection{Logic and Concept}
The Euler method is the simplest explicit first-order numerical scheme. It approximates the solution by taking a linear step forward using the slope at the current point. The method is defined by the iterative formula:
\begin{equation}
\mathbf{y}_{n+1} = \mathbf{y}_n + h \cdot f(t_n, \mathbf{y}_n)
\end{equation}
Although easy to implement, its local truncation error is proportional to $h^2$ and its global error is proportional to $h$, making it the least accurate of the methods tested. It is also only conditionally stable, requiring a small step size to avoid divergence.

\subsection{Results}
\begin{figure}[htbp]
\centerline{\includegraphics[width=\columnwidth]{Euler_Method.png}}
\caption{Model solution using Euler Method.}
\label{fig:euler}
\end{figure}

\section{Physics-Informed Neural Networks (PINNs) Method [4]}
\subsection{Logic and Concept}
In addition to classical numerical solvers, we implemented a machine learning-based approach using Physics-Informed Neural Networks (PINNs) to solve the apoptosis ODE system. PINNs are a class of neural networks that incorporate the underlying physical laws (in this case, ODEs) directly into the loss function during training. This allows the network to learn a continuous differentiable function that approximates the solution over the entire time domain.

The PINN is trained by minimizing a composite loss function that penalizes both the deviation from the ODE system (residual loss) and the mismatch with initial conditions (initial loss). The network takes time $t$ as input and outputs the predicted concentrations for all six variables. Automatic differentiation is used to compute the derivatives required for the ODE residuals.

This approach is mesh-free and can generalize well, even with sparse data, making it a promising tool for both forward and inverse problems in systems biology. The results are visualized in a graph that shows the difference between the PINN and RK4 solutions, highlighting how closely the PINN can approximate the classical numerical method across the time domain.

\subsection{Results}
\begin{figure}[htbp]
\centerline{\includegraphics[width=\columnwidth]{PINN_vs_RK4.png}}
\caption{Comparison between PINN and RK4 solutions for the apoptosis model.}
\label{fig:pinn_vs_rk4}
\end{figure}

\section{Future Work}
This comparative study of classical numerical methods and machine learning-based PINNs provides a strong foundation for further exploration. A key area for future work is the extension of the PINN framework to solve inverse problems, such as estimating unknown biological parameters (e.g., the $a_i$ constants) from sparse experimental data. Additionally, exploring hybrid approaches that combine the strengths of traditional solvers and neural networks could provide valuable insight into the evolving landscape of computational biology.

\section{Simulation Results and Discussion}
Table \ref{tab:variable_errors} reveals that while RK4 maintains relatively low errors across all variables (peaking at 1.956\% for y\_p300), the Explicit Euler method shows remarkably small errors (on the order of $10^{-6}$\%), and PINN exhibits higher but still reasonable errors (peaking at 4.34\% for y\_p53). 

\vspace{\bigskipamount}


\begin{table}[htbp]
\centering
\caption{Percentage Errors for Each Variable Across Methods}
\begin{tabular}{lccc}
\toprule
\textbf{Variable} & \textbf{RK4} & \textbf{Explicit Euler} & \textbf{PINN} \\
\midrule
Y\textsubscript{hif}  & 0.0000144\%     & 0.0000705\%     & 0.7435\%     \\
Y\textsubscript{o2}   & 0.0000047\%    & 0.0000463\%     & 0.5417\%     \\
Y\textsubscript{p300} & 0.0000488\%     & 0.000192\%      & 1.7866\%    \\
Y\textsubscript{p53}  & 0.0000579\%     & 0.0000410\%     & 1.6647\%     \\
Y\textsubscript{casp} & 0.00000903\%   & 0.0000869\%     & 0.5925\%     \\
Y\textsubscript{kp}   & 0.0000477\%     & 0.0000216\%     & 0.1722\%     \\
\bottomrule
\end{tabular}
\label{tab:variable_errors}
\end{table}

Table \ref{tab:performance_comparison} highlights the trade-offs between computational efficiency and accuracy. RK4 emerges as the most efficient method with both rapid execution (0.939 sec) and good accuracy (max error 0000579\%). The PINN approach, while offering the advantage of being mesh-free, requires significantly more computational time (0.01092  sec) and has higher errors compared to traditional numerical methods.

\begin{table}[htbp]
\centering
\caption{Overall Performance Comparison of Methods}
\begin{tabular}{lccc}
\toprule
\textbf{Method} & \textbf{Execution Time} & \textbf{Max Error} & \textbf{Min Error} \\
\midrule
RKF45 & 0.042 sec & ---- & ---- \\
RK4 & 0.939 sec & 0.0000579\% & 0.0000047\% \\
Euler & 0.189 sec & 0.000192\% & 0.0000216\% \\
PINNs & 0.01092 sec & 1.7866\% & 0.1722\% \\
\bottomrule
\end{tabular}
\label{tab:performance_comparison}
\end{table}

The quantitative results in Table~\ref{tab:time_point_comparison} show that all four methods—RKF45, RK4, Euler, and the machine learning-based PINN (ML)—produce very similar solution values for each variable at the selected time points ($t=25, 50, 75, 100$). Tables \ref{tab:variable_errors} and \ref{tab:performance_comparison} provide additional insights into the performance characteristics of each method.

\vspace{\bigskipamount}
\begin{table}[htbp]
\centering
\caption{Comparison of solution values at different time points (Initial condition: [1,0,0,0,0,0])}
\begin{tabular}{lccccccc}
\toprule
\textbf{Method} & \textbf{Time} & \textbf{y\_hif} & \textbf{y\_o2} & \textbf{y\_p300} & \textbf{y\_p53} & \textbf{y\_casp} & \textbf{y\_kp} \\
\midrule
RKF45 & 25  & 0.5407 & 2.6699 & 0.4732 & 0.4587 & 1.2789 & 0.5693 \\
      & 50  & 0.5037 & 2.8410 & 0.5773 & 0.4941 & 1.4711 & 0.5266 \\
      & 75  & 0.4996 & 2.8618 & 0.5950 & 0.4968 & 1.4943 & 0.5223 \\
      & 100 & 0.4991 & 2.8646 & 0.5978 & 0.4970 & 1.4968 & 0.5219 \\
\midrule
RK4   & 25  & 0.5420 & 2.6645 & 0.4706 & 0.4576 & 1.2735 & 0.5706 \\
      & 50  & 0.5038 & 2.8407 & 0.5771 & 0.4940 & 1.4708 & 0.5267 \\
      & 75  & 0.4996 & 2.8618 & 0.5950 & 0.4968 & 1.4943 & 0.5223 \\
      & 100 & 0.4991 & 2.8646 & 0.5978 & 0.4970 & 1.4968 & 0.5219 \\
\midrule
Euler & 25  & 0.5419 & 2.6647 & 0.4707 & 0.4576 & 1.2737 & 0.5705 \\
      & 50  & 0.5038 & 2.8408 & 0.5771 & 0.4940 & 1.4709 & 0.5267 \\
      & 75  & 0.4996 & 2.8618 & 0.5950 & 0.4968 & 1.4943 & 0.5223 \\
      & 100 & 0.4991 & 2.8646 & 0.5978 & 0.4970 & 1.4968 & 0.5219 \\
\midrule
ML    & 25  & 0.5469 & 2.6449 & 0.4661 & 0.4501 & 1.2718 & 0.5664 \\
      & 50  & 0.5081 & 2.8231 & 0.5672 & 0.4841 & 1.4567 & 0.5278 \\
      & 75  & 0.4961 & 2.8538 & 0.5916 & 0.4933 & 1.4908 & 0.5188 \\
      & 100 & 0.5027 & 2.8491 & 0.4887 & 0.4887 & 1.4878 & 0.5228 \\
\bottomrule
\end{tabular}
\label{tab:time_point_comparison}
\end{table}
\vspace{\bigskipamount}
\vspace{\bigskipamount}
\vspace{\bigskipamount}
\vspace{\bigskipamount}
\vspace{\bigskipamount}
\vspace{\bigskipamount}
\vspace{\bigskipamount}
\vspace{\bigskipamount}
\vspace{\bigskipamount}




\section{Conclusion}
We implemented and compared the Explicit Euler, 4th-Order Runge-Kutta, adaptive Runge-Kutta-Fehlberg (RKF45), and Physics-Informed Neural Network (PINN) methods for solving an ODE model of apoptosis. Using our RKF45 implementation as a high-precision baseline, we found that all four methods produced stable and consistent results. Both the 4th-Order Runge-Kutta and PINN offered a good balance of accuracy and efficiency. Overall, our results show that even simple numerical methods and modern machine learning approaches can reliably solve well-behaved biological systems.\newline

\begin{thebibliography}{00}

\bibitem{bib:euler}
L.~Euler, \textit{Institutionum calculi integralis}, 1768–1770.  
Introduced the explicit Euler method, the first-order ODE solver.

\bibitem{bib:runge_kutta}
C.~Runge, “Über die numerische Auflösung von Differentialgleichungen,” \textit{Mathematische Annalen}, vol.~46, pp. 167–178, 1895.  
M.~Kutta, “Beitrag zur näherungsweisen Integration totaler Differentialgleichungen,” \textit{Zeitschrift für Mathematik und Physik}, vol.~46, pp. 435–453, 1901.  
Together, they developed the RK family, including the 4th-order RK4 method.

\bibitem{bib:rkf45}
E.~Fehlberg, “Low-order classical Runge-Kutta formulas with step-size control and their application to some heat transfer problems,” \textit{NASA Technical Report R-315}, 1969.  
This paper introduced the adaptive RKF45 solver with embedded error control.

\bibitem{bib:pinn}
M.~Raissi, P.~Perdikaris, and G.~E.~Karniadakis, “Physics-Informed Neural Networks: A Deep Learning Framework for Solving Forward and Inverse Problems Involving Nonlinear Partial Differential Equations,” \textit{Journal of Computational Physics}, vol.~378, pp. 686–707, 2019.  
Proposed PINNs—a modern ML-based solver that embeds physical laws into neural networks.

\bibitem{bib:schiesser}
W.~E.~Schiesser, \textit{Differential Equation Analysis in Biomedical Science and Engineering: Ordinary Differential Equation Applications with R}. Wiley, 2014.  
Demonstrated biomedical modeling using ODEs and R, with case studies including apoptosis.

\end{thebibliography}

\end{document}